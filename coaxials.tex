\documentclass[11pt, a4paper]{scrartcl}

\usepackage[english]{babel}
\usepackage[paper=a4paper,bottom=3cm,top=3cm,left=2.5cm,right=2.5cm]{geometry}
\usepackage[T1]{fontenc}
\usepackage[utf8]{luainputenc}

\usepackage{graphicx}
\usepackage{wrapfig}

\usepackage{epstopdf}
\usepackage{amsmath}
\usepackage{amsfonts}

\usepackage{biblatex}
\usepackage{csquotes}

\title{An Overview Of Coaxial Rotors In Helicopter Design}
\author{Bjarne Oldenburg}
\date{\today}

\begin{document}

\maketitle

\begin{center}
    Themenausarbeitung im Rahmen der Schlüsselqualifikation eines\\ \emph{\LaTeX-Einführungskures}\\ an der Universität Stuttgart.
    \vspace{1cm}\\
\end{center}

\section{Abstract}
Most helicopter models, both contemporary and historical, use a main and rear rotor configuration, in which the torque produced by the engine is counteracted
by the rear rotor. However, there exist a few models - most notably those of the Soviet/Russian Kamov Design Bureau - that use a coaxial rotor configuration instead.While this does offer benefits in engine utilization and flight behaviour, it also adds a considerable technical challenge in rotor assembly design.

\section{Introduction}
A helicopter, named so by Ponton d' Amécourt, who derived the name from the greek elikoeioas, meaning winding (the word helix is derived from the very same origin), and pteron, which means wing, is - as implied by the etymological background - defined as an aircraft using rotating airfoils as means of generating lift, thrust and any control inputs, and that is capable of controlled hover without forward movement. The historical development of helicopters settled, after a large variety of designs in the early 1900s, on most helicopters using a larger main rotor, which generates vertical thrust and control inputs, combined with a smaller rear rotor to counteract the torque exerted on the helicopter. testing 2

\section{Conclusion}
If everything works fine, your LaTeX setup is working perfectly!

\end{document}